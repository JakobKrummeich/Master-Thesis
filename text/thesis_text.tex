\documentclass[DIV=9,11pt,BCOR=5mm,twoside=semi,abstract]{scrreprt}

\usepackage[T1]{fontenc}
\usepackage[utf8]{inputenc}
\usepackage{amsmath}
\usepackage{amssymb}
\usepackage{booktabs}
\usepackage{dsfont}
\usepackage[main=american,australian,ngerman,british]{babel}
\usepackage{csquotes}
%\usepackage[onehalfspacing]{setspace}

\usepackage[separate-uncertainty=true]{siunitx}
\sisetup{range-phrase=-,range-units=single}
\usepackage{xcolor}
\usepackage{xpatch}
\usepackage{float}
\usepackage{placeins}
\usepackage{graphicx}
\usepackage{epstopdf}

\usepackage{multirow}
\usepackage{microtype}


\usepackage[style=numeric,backend=biber, dateabbrev=false,urldate=long,sorting=none]{biblatex} 
\addbibresource{bibBA.bib}

%allow linebreaks after numbers in urls and dois 
\setcounter{biburlnumpenalty}{9000}

%\KOMAoptions{DIV=last}

%%%%%%%%%%%%%%%%%%%%%%%%%%%%%%%%%%%%%%%%%%%%%%%%%%%%%%%%%%%%%%%%%%%%%%%%%%%%%%%%%%%%%%%%%%%%%%%%%%%%%%%%%%%%%%%%%%%%%%%%%%%%%%%%%%%%%%%%%%%%%%%%%

\begin{document}
	
%	\titlehead{{\Large Ruprecht-Karls-Universität Heidelberg}}
%	\subject{Bachelor Thesis}
%	\title{Development of a cooling system for mirrors in an UHV chamber}
%	\author{Jakob Krummeich}
%	\date{13.09.2019}
%	\publishers{Adviser and first examiner: Priv.-Doz. Dr. José Ramón Crespo López-Urrutia\\
%				Second examiner: Prof. Dr. Thomas Pfeifer}
%	\maketitle

\begin{titlepage}
	\centering
	\textbf{\Large Theoretische Physik II: Soft Matter\\}
	\textbf{\Large Heinrich-Heine-Universität Düsseldorf\\}
	\vspace{12 cm}
	Master thesis in physics submitted by\\
	\vspace{4mm}
	\textbf{\large Jakob Krummeich\\}
	\vspace{4mm}
	and written at the\\
	\vspace{4mm}
	\textbf{Institute for Theoretical Physics\\}
	\vspace{4mm}
	\textbf{Düsseldorf\\}
	\vspace{4mm}
	\textbf{XX.XX.2022}
	
	\cleardoubleoddpage
	\thispagestyle{empty}
	\centering
	\textbf{\huge Critical behavior of a 2-D Lennard-Jones mixture under shear \\}
	\vspace{4mm}
	\textbf{\huge \\}
	
	\vspace{12cm}
	
	\begin{tabular}{ll}
		Adviser and first examiner: & Prof. Dr. J. Horbarch\\
		\rule{0pt}{4ex}
		Second examiner: & Prof. Dr. H. Löwen
	\end{tabular}
	
	
\end{titlepage}

%%%%%%%%%%%%%%%%%%%%%%%%%%%%%%%%%%%%%%%%%%%%%%%%%%%%%%%%%%%%%%%%%%%%%%%%%%%%%%%%%%%%%%%%%%%%%%%%%%%%%%%%%%%%%%%%%%%%%%%%%%%%%%%%%%%%%%%%%%%%%%%%%

	\begin{abstract}
	
	\end{abstract}
	
%%%%%%%%%%%%%%%%%%%%%%%%%%%%%%%%%%%%%%%%%%%%%%%%%%%%%%%%%%%%%%%%%%%%%%%%%%%%%%%%%%%%%%%%%%%%%%%%%%%%%%%%%%%%%%%%%%%%%%%%%%%%%%%%%%%%%%%%%%%%%%%%%
	
	\AfterTOCHead[toc]{%
		\thispagestyle{empty}%
		\pagestyle{empty}%
	}
	
	\tableofcontents
	\thispagestyle{empty}
	\cleardoubleoddpage
	\pagenumbering{arabic}
	
%%%%%%%%%%%%%%%%%%%%%%%%%%%%%%%%%%%%%%%%%%%%%%%%%%%%%%%%%%%%%%%%%%%%%%%%%%%%%%%%%%%%%%%%%%%%%%%%%%%%%%%%%%%%%%%%%%%%%%%%%%%%%%%%%%%%%%%%%%%%%%%%%

	\chapter{Introduction}
	\label{chap:introduction}


%%%%%%%%%%%%%%%%%%%%%%%%%%%%%%%%%%%%%%%%%%%%%%%%%%%%%%%%%%%%%%%%%%%%%%%%%%%%%%%%%%%%%%%%%%%%%%%%%%%%%%%%%%%%%%%%%%%%%%%%%%%%%%%%%%%%%%%%%%%%%%%%%
	
	\chapter{Theoretical background}
	\label{chap:theory}

	\section{Pair correlation function }
	\label{sec:theory_pair_correlation_function}
	
	The partial pair correlation functions for a binary mixture can be defined as 
	
	\begin{align}
		g_{\alpha\beta}\left( \vec{r} \right) = \frac{V}{N_{\alpha}N_{\beta}} \langle \sum_{i \neq j} \delta \left( \vec{r} - \left( \vec{r}_i - \vec{r}_j \right) \right) \rangle
	\end{align}
	
	where $\alpha,\beta \in \{A,B\}$ label the particle type, $V$ is the total volume, $N_{\alpha}$, $N_{\beta}$ denote the particle numbers, $\langle . \rangle$ indicates the ensemble average and $\delta (.)$ is the delta function. The sum runs over all pairs of particles $i,j$ where $i \neq j$, so that $\vec{r}_{ij} = \vec{r}_i - \vec{r}_j$ are the distance vectors between particle types $\alpha$ and $\beta$. Thus, $g_{\alpha \beta} \left( \vec{r} \right)$ simply is the averaged distribution of  $\vec{r}_{ij}$ over the volume space $V$. Note that this definition implies $g_{\alpha\beta} (\vec{r}) = g_{\alpha\beta} (-\vec{r})$, i.e. $g_{\alpha\beta}(\vec{r})$ is an even function. \par
	
	In 2 dimensions with the usual map
	
	\begin{align}
		\vec{r} = 
		\begin{pmatrix}
			r \cdot \cos \phi\\
			r \cdot \sin \phi
		\end{pmatrix}
	\end{align}

	for $r\in (0, \infty]$, $\phi \in [0, 2\pi)$ and
	
	\begin{align}
		\phi_{ij} = \sphericalangle \left( \vec{r}_i,\vec{r}_j \right)
	\end{align}
	
	we get 
	
	\begin{align}
		g_{\alpha\beta}\left( \vec{r} \right) = \frac{2V}{N_{\alpha}N_{\beta}} \langle \sum_{i} \sum_{i < j} \frac{1}{r} \delta \left( r - | \vec{r}_i - \vec{r}_j | \right) \delta \left( \phi - \phi_{ij} \right) \rangle.
	\end{align}

	To examine structural anisotropy we expand $g_{\alpha\beta}(\vec{r})$ into spherical harmonics: 
	
	\begin{align}
		g_{\alpha\beta}(\vec{r}) = \sum_{l=0}^{\infty} \sum_{m=-l}^{l} g_{lm}^{\alpha\beta} Y_{lm}(\theta, \phi)
	\end{align}

	where $Y_{lm}(\theta, \phi)$ are spherical harmonics with degree $l$ and order $m$. In 2 dimensions we restrict $\theta = \pi/2$. Since the spherical harmonics form a complete orthonormal set the coefficients $g^{\alpha\beta}_{lm}$ are given by
	
	\begin{align}
		g^{\alpha\beta}_{lm} (r) &= \frac{2V}{N_{\alpha}N_{\beta}} \langle \sum_{i} \sum_{i < j} \int_{0}^{2\pi} \text{d}\phi' \frac{1}{r} \delta \left( r - | \vec{r}_i - \vec{r}_j | \right) \delta \left( \phi' - \phi_{ij} \right) Y_{lm}^{*}(\pi/2,\phi') \rangle \\
		&= \frac{2V}{N_{\alpha}N_{\beta}} \langle \sum_{i} \sum_{i < j} \frac{1}{r} \delta \left( r - | \vec{r}_i - \vec{r}_j | \right)  Y_{lm}^{*}(\pi/2,\phi_{ij}) \rangle.
	\end{align}

	Note that $g_{lm}^{\alpha\beta}$ depends only on $r$ since the angular dependence of $g_{\alpha\beta}$ is fully included in $Y_{lm}$. Furthermore, since $g_{\alpha\beta}(\vec{r}) = g_{\alpha\beta} (-\vec{r} )$ and $Y_{lm}(-\vec{r}) = (-1)^{l} Y_{lm}(\vec{r})$ we see that all $g_{lm}^{\alpha\beta}$ with odd $l$ vanish. For isotropic systems all coefficients except $g_{00}^{\alpha\beta}(r)$ vanish. \par
	
	Of special importance in the context of Couette flow is the imaginary part of $g_{22}^{\alpha\beta}$ as it can be linked to the configurational part of the shear stress. In Cartesian coordinates we have 
	
	\begin{align}
		\text{Im}Y_{22} = \sqrt{\frac{15}{8\pi}} \frac{xy}{r^2}
	\end{align}

	so that in 2 dimensions we get
	
	\begin{align}
		\text{Im} g_{22}^{\alpha\beta} = \sqrt{\frac{15}{2\pi}} \frac{V}{N_{\alpha} N_{\beta}} \langle \sum_{i} \sum_{i < j} \frac{1}{r^3} \delta \left( r - | \vec{r}_i - \vec{r}_j | \right) (x_i - y_j)(y_i - y_j) \rangle.
	\end{align}

%%%%%%%%%%%%%%%%%%%%%%%%%%%%%%%%%%%%%%%%%%%%%%%%%%%%%%%%%%%%%%%%%%%%%%%%%%%%%%%%%%%%%%%%%%%%%%%%%%%%%%%%%%%%%%%%%%%%%%%%%%%%%%%%%%%%%%%%%%%%%%%%%

	\section{Static structure factors}
	\label{sec:theory_structure_factors}
	
	We can define partial structure factors by
	
	\begin{align}
		S_{\alpha\beta}(\vec{q}) = \frac{1}{N} \langle \sum_{i,j,i \neq j} \exp\left[ -\text{i} \vec{q} (\vec{r}_i - \vec{r}_j) \right] \rangle
	\end{align}

	where the sum runs over all distance vectors $\vec{r}_{ij} = \vec{r}_i - \vec{r}_j$ of particles of types $\alpha,\beta \in \{A,B\}$. Thus, the partial structure factors are proportional to the Fourier transform of the partial pair correlation functions $g_{\alpha\beta}(\vec{r})$ given in section \ref{sec:theory_pair_correlation_function}. Furthermore, with this definition it is clear that $S_{\alpha\beta} (\vec{q}) \in \mathds{R}$ as the partial pair correlation functions are even. \par
	
	In a system with periodic boundary conditions we need to respect the symmetry, so that only a discrete set of $\vec{q}$ values is allowed. In a 2 dimensional system in a quadratic box with length $L$ these are given by 
	
	\begin{align}
		\vec{q} \in 
		\begin{pmatrix}
			\frac{2\pi}{L} z_x\\
			\frac{2\pi}{L} z_y
		\end{pmatrix}, 
		z_{x,y} \in \mathds{Z}.
	\end{align}


%%%%%%%%%%%%%%%%%%%%%%%%%%%%%%%%%%%%%%%%%%%%%%%%%%%%%%%%%%%%%%%%%%%%%%%%%%%%%%%%%%%%%%%%%%%%%%%%%%%%%%%%%%%%%%%%%%%%%%%%%%%%%%%%%%%%%%%%%%%%%%%%%

	\chapter{Model description}
	\label{chap:model_description}

%%%%%%%%%%%%%%%%%%%%%%%%%%%%%%%%%%%%%%%%%%%%%%%%%%%%%%%%%%%%%%%%%%%%%%%%%%%%%%%%%%%%%%%%%%%%%%%%%%%%%%%%%%%%%%%%%%%%%%%%%%%%%%%%%%%%%%%%%%%%%%%%%	

	\chapter{Simulation methods}
	\label{chap:simulation_methods}
	
	\section{Velocity-Verlet-Algorithm}
	\label{sec:velocity_verlet_algorithm}
	
	To numerically solve the equation of motion for Hamiltonian systems we use the Velocity-Verlet-Algorithm, which evolves the position $\vec{r}_i(t)$ and velocity $\vec{v}_i(t)$ of particle $i$ from time $t$ to $t+\Delta t$ according to
	
	\begin{align}
		\vec{r}_i (t+\Delta t) &= \vec{r}_i (t) + \Delta t \cdot \vec{v}_i (t) + \frac{({\Delta t})^2}{2m_i}  \vec{F}_i(t) \\
		\vec{v}_i (t+\Delta t) &= \vec{v}_i (t) + \frac{\Delta t}{2 m_i} \left( \vec{F}_i(t) + \vec{F}_i (t+ \Delta t) \right)
	\end{align}
	
	where $m_i$ is the particle's mass, $\vec{F}_i$ is the force on the particle at time $t$ or $t + \Delta t$ respectively. Note that to use this algorithm conveniently it is necessary that $\vec{F}_i (t + \Delta t)$ does not depend on the velocities of the particles, which is the case for our model since the forces are conservative. \par 
	¸This algorithm has the property to be time-reversible and conserving the total energy up to corrections of order $(\Delta t)^2$.
	
	\section{Lees-Edwards boundary conditions}
	\label{sec:lees_edwards_boundary_conditions}
	
	Lees-Edwards boundary conditions have been introduced in 1972 by A.W. Lees and S.F. Edwards as a method to introduce shear in a simulation box with periodic boundary conditions [CITATION NEEDED]. In 'usual' periodic boundary conditions we can think of  the simulation box to be surrounded by an infinite number of copies of itself, which are also called 'image' boxes in this context. To introduce shear in the system using Lees-Edwards boundary conditions we 'slide' the image boxes with a constant speed against the 'original' box.
	
	\begin{figure}[H]
		\centering
		\includegraphics[trim =0cm 0cm 0cm 0cm,clip,width = 0.6\textwidth]{images/periodic_boundary_conditions_1}
		\caption{'Usual' periodic boundary conditions}
		\label{fig:periodic_boundary_conditions}
	\end{figure}
%%%%%%%%%%%%%%%%%%%%%%%%%%%%%%%%%%%%%%%%%%%%%%%%%%%%%%%%%%%%%%%%%%%%%%%%%%%%%%%%%%%%%%%%%%%%%%%%%%%%%%%%%%%%%%%%%%%%%%%%%%%%%%%%%%%%%%%%%%%%%%%%%

	\chapter{Results}
	\label{chap:results}

%%%%%%%%%%%%%%%%%%%%%%%%%%%%%%%%%%%%%%%%%%%%%%%%%%%%%%%%%%%%%%%%%%%%%%%%%%%%%%%%%%%%%%%%%%%%%%%%%%%%%%%%%%%%%%%%%%%%%%%%%%%%%%%%%%%%%%%%%%%%%%%%%

\chapter{Conclusion and Outlook}
\label{chap:conclusion}

%%%%%%%%%%%%%%%%%%%%%%%%%%%%%%%%%%%%%%%%%%%%%%%%%%%%%%%%%%%%%%%%%%%%%%%%%%%%%%%%%%%%%%%%%%%%%%%%%%%%%%%%%%%%%%%%%%%%%%%%%%%%%%%%%%%%%%%%%%%%%%%%%%

	\nocite{*}
	\begin{otherlanguage}{british}	%other language ensures the date format is day/month/year and not month/day/year
		\printbibliography[title =Sources and References]
	\end{otherlanguage}

%%%%%%%%%%%%%%%%%%%%%%%%%%%%%%%%%%%%%%%%%%%%%%%%%%%%%%%%%%%%%%%%%%%%%%%%%%%%%%%%%%%%%%%%%%%%%%%%%%%%%%%%%%%%%%%%%%%%%%%%%%%%%%%%%%%%%%%%%%%%%%%%%%
	
\chapter*{Attachments}




\chapter*{Erklärung}

Ich versichere, dass ich diese Arbeit selbstständig verfasst und keine anderen als die
angegebenen Quellen und Hilfsmittel benutzt habe. \vspace{2cm}


Düsseldorf, den XX.XX.2022

\vspace{2 cm}


Jakob Krummeich


	
\end{document}